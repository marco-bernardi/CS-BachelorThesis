\chapter{Introduction}
\label{cap:Introduction}
% 
% Introduzione al contesto applicativo.\\
% 
% \noindent Esempio di citazione in linea \\
% \cite{site:agile-manifesto}. \\
% 
% \noindent Esempio di citazione nel pie' di pagina \\
% citazione\footcite{womak:lean-thinking} \\

\section{The company}

Breton S.p.A is an Italian company specializing in the manufacturing and engineering of high-tech machinery and systems for various industries. \\
Founded in 1963 by Marcello Toncelli, Breton has grown to become a global leader in the production of advanced industrial equipment. \\
The company's headquarters are located in Castello di Godego, Italy, 
and it operates multiple production facilities and subsidaries worldwide.\\
Breton is known for its cutting-edge technologies and innovative solutions, catering to sectors such as stone processing, metalworking, aerospace, automotive, and many more.
Breton's product portfolio encompasses a wide range of machinery, meeting the demands of both small-scale workshops and large industrial enterprises.\\
With a strong emphasis on research and development, Breton has made significant contributions to the advancement of automation and digitalization in various industries. The company continually invests in cutting-edge technologies, such as artificial intelligence, Internet of Things (\gls{iotg}), and machine learning, to provide state-of-the-art solutions to its customers.
Furthermore, Breton places a high value on sustainability and eco-friendly practices. The company focuses on developing energy-efficient machinery, reducing waste generation, and promoting sustainable manufacturing processes.
\subsection{Products and Services}
\subsubsection{Products}
Breton's product portfolio encompasses a wide range of machinery, including:
\begin{itemize}
    \item \textbf{CNC machines:} Breton's \gls{cncg} machines are designed to provide high precision and accuracy in machining operations. The company offers a wide range of CNC machines, including vertical machining centers, horizontal machining centers, and 5-axis machining centers.
    \item \textbf{Cutting and shaping systems:} Breton's cutting and shaping systems are designed to provide high precision and accuracy in cutting and shaping operations. The company offers a wide range of cutting and shaping systems, including waterjet cutting systems, laser cutting systems, plasma cutting systems, and wire cutting systems.
    \item \textbf{Polishing equipment:} Breton's polishing equipment is designed to provide high precision and accuracy in polishing operations. The company offers a wide range of polishing equipment, including polishing machines, polishing robots, and polishing systems.
    \item \textbf{Waterjet cutting systems:} Breton's waterjet cutting systems are designed to provide high precision and accuracy in waterjet cutting operations. The company offers a wide range of waterjet cutting systems, including waterjet cutting machines, waterjet cutting robots, and waterjet cutting systems.
    \item \textbf{Robotic solutions:} Breton's robotic solutions are designed to provide high precision and accuracy in robotic operations. The company offers a wide range of robotic solutions, including robotic arms, robotic cells, and robotic systems.
    \item \textbf{Software:} Breton provides software solutions for the management of the productivity process, that can be fully integrated with the existing systems, for grant the best performance of the machinery and grant the control over the production plant.
\end{itemize} 
\subsubsection{Services}
Breton offers a wide range of services to its customers, including:
\begin{itemize}
    \item \textbf{Consulting:} Breton provides consulting services to its customers, helping them choose the right machinery for their needs. The company's experts analyze the customer's requirements and recommend the most suitable solutions.
    \item \textbf{Installation:} Breton's technicians install the machinery at the customer's site and ensure that it is functioning properly.
    \item \textbf{Training:} Breton offers training programs to its customers, teaching them how to operate the machinery and get the most out of it.
    \item \textbf{Maintenance:} Breton provides maintenance services to its customers, ensuring that the machinery is running smoothly and efficiently.
\end{itemize}
\subsection{Certifications}
Breton is continuously improving its products, services and workflows to meet the highest standards of quality, safety, and environmental protection. \\
The company is certified according to the following standards:
\begin{itemize}
    \item \textbf{ISO 9001:} Breton is certified according to the ISO 9001 standard, which specifies requirements for a quality management system.
    \item \textbf{ISO 14001:} Breton is certified according to the ISO 14001 standard, which specifies requirements for an environmental management system.
    \item \textbf{UNI INAIL ed. 2001:} Breton is certified according to the UNI INAIL ed. 2001 standard, which specifies requirements for a health and safety management system.
\end{itemize}
\section{The idea}
The idea born from a company request, they want to extract the main features of a marble slab, such as the veins, texture, and colors for automatically create a digital fingerprint of the marble slab.\\
This digital fingerprint can be easily achieved with a classic \gls{mlg} model. Veins in marble slabs are not always the same, and the marble slabs are not always perfect, so teach to a \gls{mlg} model what is and what is not a veins could be very challenging.\\
For this reason it's needed a lot of images with the respective veins path for train the model, and this is not always possible for different reasons:
\begin{itemize}
    \item Breton work only on test slabs, so they don't have a large amount of images. 
    For solve this problem, they've to rely on the customers, and many times the customers don't want to share their images.
    \item Manually extract the veins path from the images is very time consuming, and will take a lot of time and human resources.
\end{itemize}
So the idea is to create a generative model that can generate a large amount of images with the respective veins path, and use this images for train the \gls{mlg} model.\\
\subsection{Side Idea}
From the main idea, born a side idea, that is to create a \gls{cncg} program from an image file.\\
In this way they can recreate veins on technological stones slabs, by the use of a \gls{cncg} machine.\\
This idea is very challenging, because convert an image file to a \gls{cncg} program is not a simple task, \gls{cncg} program is a list of instructions for the machine, and the image file is a matrix of pixels.\\
For this reason, the side idea was reverted and become: "Generate a marble slab image from a \gls{cadg} model".\\
The \gls{gang} model need to be able to reproduce colors, textures and the veins should follow the path of the \gls{cadg} model.\\
This is more feasible, the \gls{cadg} model can be easily converted to a \gls{cncg} program, and the \gls{cncg} program can be used for create a marble slab.\\

%% OLD IDEA
%The idea of the project was born from a request of some customers, that want to extract the main features of a marble slab, such as the veins, texture, and colors for creating a digital
%fingerprint of the marble slab.\\
%In this way the customer can have a digital copy of the marble slab, and can use it for create a \gls{cadg} model of the marble slab, and use it for create a \gls{cncg} program for the machines.\\
%Having a model capable of extract the main features of the marble slab, could reduce the workload of the designer, that can avoid to do it by hand.\\
%This request was a bit eulerian, because the veins are not always the same and the marble slab are not always perfect, so teach to a \gls{mlg} model what is and what is not a veins could be very challenging.\\
%Plus convert an image file to a \gls{cncg} program is not a simple task, because the \gls{cncg} program is a list of instructions for the machine, and the image is a matrix of pixels, so the conversion is not trivial.\\
%
%Then it was choose to revert the problem, and instead of convert an image to a \gls{cncg} program, it was choose to convert a \gls{cadg} file to an image, and the relative \gls{cncg} program.\\
%The \gls{gang} model need to be trained for reproduce the colors and the texture of the marble slab, with the given path of the veins exported from the \gls{cadg} file.\\
%This approach was more feasible, because the \gls{cadg} file can be easily converted to a \gls{cncg} program with the existing software.\\

\section{Objectives}
The objectives of the project could be classified in tree main categories recognized by the following letters:
\begin{itemize}
    \item \textbf{M:} Mandatory objectives, that are the main objectives of the project and a request of the commissioning company.
    \item \textbf{D:} Desiderable objectives, that are the secondary objectives of the project, not necessary
    for the commissioning company but that can give added value to the project.
    \item \textbf{O:} Optional objectives, that are the optional objectives of the project.
\end{itemize}
Each objectives can be identified by a unique code, that is composed by the letter of the category and a number that identify the objective.\\
The objectives are listed in the following table:

\begin{table}[H] 
    \caption{Mandatory objectives table}
    \label{tab:man-objectives}
    \centering
    \begin{tabularx}{\textwidth}{|c|X|c|}
        \hline
        \textbf{Code} & \textbf{Description} & \textbf{Category}\\
        \hline
        M1 & Get an analytic and multydisciplinar thought, thanks to the decomposition of the problem in subproblems & M\\
        \hline
        M2 & Reach a level of autonomy in the management of the project, with sinthesis and critical thinking of the problems & M\\
        \hline
        M3 & Quality on the production of technological artifacts and on the documentation & M\\
    \end{tabularx}
\end{table}

\begin{table}[H] 
    \caption{Desiderable objectives table}
    \label{tab:des-objectives}
    \centering
    \begin{tabularx}{\textwidth}{|c|X|c|}
        \hline
        \textbf{Code} & \textbf{Description} & \textbf{Category}\\
        \hline
        D1 & Development of a \gls{pocg} for generating marble images with a \gls{gang} model or similar technology & D\\
        \hline
        D2 & Development of a \gls{pocg} for augmenting resolution of an image & D\\
        \hline
        D3 & Validation of the artifacts produced & D\\
    \end{tabularx}
\end{table}

\begin{table}[H] 
    \caption{Optional objectives table}
    \label{tab:opt-objectives}
    \centering
    \begin{tabularx}{\textwidth}{|c|X|c|}
        \hline
        \textbf{Code} & \textbf{Description} & \textbf{Category}\\
        \hline
        O1 & First product engineering approaches developed & O\\
        \hline
        O2 & Product testing in a manufacturing production environment & O\\
    \end{tabularx}
\end{table}
\section{Text structure}

\begin{description}
    \item[{\hyperref[cap:Introduction]{First chapter}}] 
    The first chapter introduce the company and its products and services.\\
    It also describes the idea of the project and the text structure.

    \item[{\hyperref[cap:descrizione-stage]{Second chapter}}] The second chapter describes the process and the methodology used for the development of the project.
    
    \item[{\hyperref[cap:process-methodologies]{Third chapter}}] Third chapter describes the internship experience, in all its aspects.
    
    \item[{\hyperref[cap:design-coding]{Fifth chapter}}] The fifth chapter describes the design and the implementation of the project.
    It describes the technologies used and the architecture of the project.
    
\end{description}

During the writing of this document, the following conventions were adopted:
\begin{itemize}
    \item acronyms, abbreviations and ambiguous or uncommon terms mentioned are defined in the glossary, located at the end of this document;
    \item for the first occurrence of the terms defined in the glossary, the following nomenclature is used: \emph{word}\glsfirstoccur;
	\item the terms that require further explanation like technical words are marked with a subscript \emph{italic};
\end{itemize}

\chapter{Introduction}
\label{cap:Introduction}

Introduzione al contesto applicativo.\\

\noindent Esempio di citazione in linea \\
\cite{site:agile-manifesto}. \\

\noindent Esempio di citazione nel pie' di pagina \\
citazione\footcite{womak:lean-thinking} \\

\section{The company}

Breton S.p.A is an Italian company specializing in the manufacturing and engineering of high-tech machinery and systems for various industries. \\
Founded in 1963 by Marcello Toncelli, Breton has grown to become a global leader in the production of advanced industrial equipment. \\
The company's headquarters are located in Castello di Godego, Italy, 
and it operates multiple production facilities and subsidaries worldwide.\\
Breton is known for its cutting-edge technologies and innovative solutions, catering to sectors such as stone processing, metalworking, aerospace, automotive, and many more.
Breton's product portfolio encompasses a wide range of machinery, including \gls{cncg} machines, cutting and shaping systems, polishing equipment, waterjet cutting systems, and robotic solutions. These machines are designed to enhance productivity, precision, and efficiency in manufacturing processes, meeting the demands of both small-scale workshops and large industrial enterprises.\\
With a strong emphasis on research and development, Breton has made significant contributions to the advancement of automation and digitalization in various industries. The company continually invests in cutting-edge technologies, such as artificial intelligence, Internet of Things (\gls{iotg}), and machine learning, to provide state-of-the-art solutions to its customers.
Furthermore, Breton places a high value on sustainability and eco-friendly practices. The company focuses on developing energy-efficient machinery, reducing waste generation, and promoting sustainable manufacturing processes.
\section{The idea}

The first request of the company was to extract the main features of the marble slab, such as the veins for translate it to GCODE and process it with the company \gls{cncg} machines.\\
This request was a bit eulerian, because the veins are not always the same and the marble slab are not always perfect; plus generate GCODE from an image is not a simple task.\\
So the idea become to generate a marble slab image from a \gls{cadg} designed image with the tracks of the veins, with all the main features of the marble slab.\\
In this way the GCODE it's already known, because it was designed by the \gls{cadg} software, and the marble slab image was generated by the \gls{gang}.\\

\section{Text structure}

\begin{description}
    \item[{\hyperref[cap:processi-metodologie]{First chapter}}] descrive ...
    
    \item[{\hyperref[cap:descrizione-stage]{Second chapter}}] approfondisce ...
    
    \item[{\hyperref[cap:analisi-requisiti]{Il quarto capitolo}}] approfondisce ...
    
    \item[{\hyperref[cap:progettazione-codifica]{Il quinto capitolo}}] approfondisce ...
    
    \item[{\hyperref[cap:verifica-validazione]{Il sesto capitolo}}] approfondisce ...
    
    \item[{\hyperref[cap:conclusioni]{Nel settimo capitolo}}] descrive ...
\end{description}

Riguardo la stesura del testo, relativamente al documento sono state adottate le seguenti convenzioni tipografiche:
\begin{itemize}
	\item gli acronimi, le abbreviazioni e i termini ambigui o di uso non comune menzionati vengono definiti nel glossario, situato alla fine del presente documento;
	\item per la prima occorrenza dei termini riportati nel glossario viene utilizzata la seguente nomenclatura: \emph{parola}\glsfirstoccur;
	\item i termini in lingua straniera o facenti parti del gergo tecnico sono evidenziati con il carattere \emph{corsivo}.
\end{itemize}

\chapter{Introduction}
% \label{cap:Introduction}
% 
% Introduzione al contesto applicativo.\\
% 
% \noindent Esempio di citazione in linea \\
% \cite{site:agile-manifesto}. \\
% 
% \noindent Esempio di citazione nel pie' di pagina \\
% citazione\footcite{womak:lean-thinking} \\

\section{The company}

Breton S.p.A is an Italian company specializing in the manufacturing and engineering of high-tech machinery and systems for various industries. \\
Founded in 1963 by Marcello Toncelli, Breton has grown to become a global leader in the production of advanced industrial equipment. \\
The company's headquarters are located in Castello di Godego, Italy, 
and it operates multiple production facilities and subsidaries worldwide.\\
Breton is known for its cutting-edge technologies and innovative solutions, catering to sectors such as stone processing, metalworking, aerospace, automotive, and many more.
Breton's product portfolio encompasses a wide range of machinery, meeting the demands of both small-scale workshops and large industrial enterprises.\\
With a strong emphasis on research and development, Breton has made significant contributions to the advancement of automation and digitalization in various industries. The company continually invests in cutting-edge technologies, such as artificial intelligence, Internet of Things (\gls{iotg}), and machine learning, to provide state-of-the-art solutions to its customers.
Furthermore, Breton places a high value on sustainability and eco-friendly practices. The company focuses on developing energy-efficient machinery, reducing waste generation, and promoting sustainable manufacturing processes.
\subsection{Products and Services}
\subsubsection{Products}
Breton's product portfolio encompasses a wide range of machinery, including:
\begin{itemize}
    \item \textbf{CNC machines:} Breton's \gls{cncg} machines are designed to provide high precision and accuracy in machining operations. The company offers a wide range of CNC machines, including vertical machining centers, horizontal machining centers, and 5-axis machining centers.
    \item \textbf{Cutting and shaping systems:} Breton's cutting and shaping systems are designed to provide high precision and accuracy in cutting and shaping operations. The company offers a wide range of cutting and shaping systems, including waterjet cutting systems, laser cutting systems, plasma cutting systems, and wire cutting systems.
    \item \textbf{Polishing equipment:} Breton's polishing equipment is designed to provide high precision and accuracy in polishing operations. The company offers a wide range of polishing equipment, including polishing machines, polishing robots, and polishing systems.
    \item \textbf{Waterjet cutting systems:} Breton's waterjet cutting systems are designed to provide high precision and accuracy in waterjet cutting operations. The company offers a wide range of waterjet cutting systems, including waterjet cutting machines, waterjet cutting robots, and waterjet cutting systems.
    \item \textbf{Robotic solutions:} Breton's robotic solutions are designed to provide high precision and accuracy in robotic operations. The company offers a wide range of robotic solutions, including robotic arms, robotic cells, and robotic systems.
    \item \textbf{Software:} Breton provides software solutions for the management of the productivity process, that can be fully integrated with the existing systems, for grant the best performance of the machinery and grant the control over the production plant.
\end{itemize} 
\subsubsection{Services}
Breton offers a wide range of services to its customers, including:
\begin{itemize}
    \item \textbf{Consulting:} Breton provides consulting services to its customers, helping them choose the right machinery for their needs. The company's experts analyze the customer's requirements and recommend the most suitable solutions.
    \item \textbf{Installation:} Breton's technicians install the machinery at the customer's site and ensure that it is functioning properly.
    \item \textbf{Training:} Breton offers training programs to its customers, teaching them how to operate the machinery and get the most out of it.
    \item \textbf{Maintenance:} Breton provides maintenance services to its customers, ensuring that the machinery is running smoothly and efficiently.
\end{itemize}
\subsection{Certifications}
Breton is continuously improving its products, services and workflows to meet the highest standards of quality, safety, and environmental protection. \\
The company is certified according to the following standards:
\begin{itemize}
    \item \textbf{ISO 9001:} Breton is certified according to the ISO 9001 standard, which specifies requirements for a quality management system.
    \item \textbf{ISO 14001:} Breton is certified according to the ISO 14001 standard, which specifies requirements for an environmental management system.
    \item \textbf{UNI INAIL ed. 2001:} Breton is certified according to the UNI INAIL ed. 2001 standard, which specifies requirements for a health and safety management system.
\end{itemize}
\section{The idea}

The first request of the company was to extract the main features of the marble slab, such as the veins for translate it to GCODE and process it with the company \gls{cncg} machines.\\
This request was a bit eulerian, because the veins are not always the same and the marble slab are not always perfect; plus generate GCODE from an image is not a simple task.\\
So the idea become to generate a marble slab image from a \gls{cadg} designed image with the tracks of the veins, with all the main features of the marble slab.\\
In this way the GCODE it's already known, because it was designed by the \gls{cadg} software, and the marble slab image was generated by the \gls{gang}.\\

\section{Text structure}

\begin{description}
    \item[{\hyperref[cap:processi-metodologie]{First chapter}}] descrive ...
    
    \item[{\hyperref[cap:descrizione-stage]{Second chapter}}] approfondisce ...
    
    \item[{\hyperref[cap:analisi-requisiti]{Il quarto capitolo}}] approfondisce ...
    
    \item[{\hyperref[cap:progettazione-codifica]{Il quinto capitolo}}] approfondisce ...
    
    \item[{\hyperref[cap:verifica-validazione]{Il sesto capitolo}}] approfondisce ...
    
    \item[{\hyperref[cap:conclusioni]{Nel settimo capitolo}}] descrive ...
\end{description}

Riguardo la stesura del testo, relativamente al documento sono state adottate le seguenti convenzioni tipografiche:
\begin{itemize}
	\item gli acronimi, le abbreviazioni e i termini ambigui o di uso non comune menzionati vengono definiti nel glossario, situato alla fine del presente documento;
	\item per la prima occorrenza dei termini riportati nel glossario viene utilizzata la seguente nomenclatura: \emph{parola}\glsfirstoccur;
	\item i termini in lingua straniera o facenti parti del gergo tecnico sono evidenziati con il carattere \emph{corsivo}.
\end{itemize}

\chapter{Introduction}\label{cap:Introduction}
\section{The company}
Breton S.p.A is an Italian company specialized in the design, engineering and production of machinery and advanced systems for various industries. 
Established in 1963 by Marcello Toncelli, Breton gained international recognition as a leader in the production of cutting-edge industrial equipment. 
The company's headquarters are located in Castello di Godego, Italy, with numerous production facilities and branches strategically located around the world. 
Breton's experience spans multiple industries, including stone working, metal working, aerospace, automotive and more.
Their extensive product portfolio includes a wide variety of machinery, catering for the needs of both small workshops and large industrial companies.
With a strong focus on research and development, Breton has been instrumental in driving advances in automation and digitization within various industries.
The company constantly invests in cutting-edge technologies such as the artificial intelligence, Internet of Things (\gls{iotg}\glsfirstoccur) and machine learning to provide cutting-edge solutions to its customers. 
Also, Breton places a significant emphasis on sustainability and environmentally friendly practices. 
Through the development of energy efficient machinery, waste reduction initiatives and the promotion of sustainability production processes, Breton actively contributes to a greener and more sustainable environment future.
\subsection{Products and Services}
\subsubsection{Products}
Breton's product portfolio encompasses a wide range of machinery, including:
\begin{itemize}
    \item \textbf{CNC machines:} Breton's \gls{cncg}\glsfirstoccur machines are designed to provide high precision and accuracy in machining operations. The company offers a wide range of \gls{cncg} machines, including vertical machining centers, horizontal machining centers, and 5-axis machining centers.
    \item \textbf{Cutting and shaping systems:} Breton's cutting and shaping systems are designed to provide high precision and accuracy in cutting and shaping operations. The company offers a wide range of cutting and shaping systems, including waterjet cutting systems, laser cutting systems, plasma cutting systems, and wire cutting systems.
    \item \textbf{Polishing equipment:} Breton's polishing equipment is designed to provide high precision and accuracy in polishing operations. The company offers a wide range of polishing equipment, including polishing machines, polishing robots, and polishing systems.
    \item \textbf{Robotic solutions:} Breton's robotic solutions are designed to provide high precision and accuracy in robotic operations. The company offers a wide range of robotic solutions, including robotic arms, robotic cells, and robotic systems.
    \item \textbf{Software:} Breton provides software solutions for the management of the productivity process, that can be fully integrated with the existing systems, for grant the best performance of the machinery and grant the control over the production plant.
\end{itemize} 
\subsubsection{Services}
Breton offers a wide range of services to its customers, including:
\begin{itemize}
    \item \textbf{Consulting:} Breton provides consulting services to its customers, helping them choose the right machinery for their needs. The company's experts analyze the customer's requirements and recommend the most suitable solutions.
    \item \textbf{Installation:} Breton's technicians install the machinery at the customer's site and ensure that it is functioning properly.
    \item \textbf{Training:} Breton offers training programs to its customers, teaching them how to operate the machinery and get the most out of it.
    \item \textbf{Maintenance:} Breton provides maintenance services to its customers, ensuring that the machinery is running smoothly and efficiently.
\end{itemize}
\subsection{Certifications}
Breton is continuously improving its products, services and workflows to meet the highest standards of quality, safety, and environmental protection. \\
The company is certified according to the following standards:
\begin{itemize}
    \item \textbf{ISO 9001:} Breton is certified according to the ISO 9001 standard, which specifies requirements for a quality management system.
    \item \textbf{ISO 14001:} Breton is certified according to the ISO 14001 standard, which specifies requirements for an environmental management system.
    \item \textbf{UNI INAIL ed. 2001:} Breton is certified according to the UNI INAIL ed. 2001 standard, which specifies requirements for a health and safety management system.
\end{itemize}
\section{The idea}
The concept originated from a request made by the company aiming to extract key characteristics of a marble slab, including its veins, texture, and colors, in order to automatically generate a digital representation, or fingerprint, of the slab. 
This digital fingerprint can be effectively created using a conventional machine learning (\gls{mlg}\glsfirstoccur) model. 
However, accurately teaching the model to distinguish between veins and other features in marble slabs poses significant challenges, as the veins can vary and the slabs themselves are not always flawless.

To address this issue, a substantial number of images depicting the veins' paths are required to train the model effectively. 
However, obtaining such images is often problematic for various reasons:
\begin{itemize}
    \item The company Breton primarily works with test slabs and thus lacks a large collection of images.
    \item Depending on customers for image contributions is not always feasible, as they may be unwilling to share their own images.
\end{itemize}
Manually extracting the veins' paths from images is an extremely time-consuming task that demands significant human resources. 
Consequently, the proposed solution involves creating a generative model capable of producing a substantial quantity of images that include the respective veins' paths. 
These generated images can then be utilized to train the machine learning model effectively.
\subsection{Side Idea}
Derived from the main concept, an ancillary idea emerged: the creation of a computer numerical control (\gls{cncg}\glsfirstoccur) program based on an image file. 
This would enable the recreation of veins on engineered stone slabs using a \gls{cncg} machine. 
However, implementing this idea presents considerable challenges, as converting an image file into a \gls{cncg} program is a complex task. 
A \gls{cncg} program consists of a sequence of instructions for the machine, while an image file is composed of a matrix of pixels.

Consequently, the ancillary idea was revised and transformed into the following: ``Generating a marble slab image from a computer-aided design (\gls{cadg}\glsfirstoccur) model``.
This approach proves more viable, as the \gls{cadg} model can be readily converted into a \gls{cncg} program, which in turn can be utilized to fabricate a marble slab.

To accomplish this, a generative adversarial network (\gls{gang}\glsfirstoccur) model is required. 
The \gls{gang} model must possess the ability to accurately reproduce colors, textures, and veins that adhere to the path defined by the \gls{cadg} model. 
This approach offers increased feasibility, as the \gls{cadg} model can be easily transformed into a \gls{cncg} program, enabling the subsequent production of a marble slab.

\section{Goals}\label{sec:goals}
The goals of the project can be categorized into three main categories denoted by the following letters:
\begin{itemize}
    \item\textbf{M:} Mandatory goals, which represent the primary objectives of the project and are explicitly required by the commissioning company.
    \item\textbf{D:} Desirable goals, which serve as secondary objectives for the project and are not essential for the commissioning company but can add value to the overall outcome.
    \item\textbf{O:} Optional goals, which are additional objectives for the project that may be pursued based on feasibility and available resources.
\end{itemize}
Each goal is assigned a unique code consisting of the category letter followed by a number that identifies the specific objective.
The goals are listed in the following table:
\begin{table}[H]
    \caption{Goals}\label{tab:goals}
    \centering
    \begin{tabularx}{\textwidth}{|c|X|c|}
        \hline
        \textbf{Code} & \textbf{Description} & \textbf{Category}\\
        \hline
        M1 & Get an analytic and multidisciplinary thought, thanks to the decomposition of the problem in sub-problems & M\\
        \hline
        M2 & Reach a level of autonomy in the management of the project, with synthesis and critical thinking of the problems & M\\
        \hline
        M3 & Quality on the production of technological artifacts and on the documentation & M\\
        \hline
        D1 & Development of a \gls{pocg}\glsfirstoccur for generating marble images with a \gls{gang} model or similar technology & D\\
        \hline
        D2 & Development of a \gls{pocg} for augmenting resolution of an image & D\\
        \hline
        D3 & Validation of the artifacts produced & D\\
        \hline
        O1 & First product engineering approaches developed & O\\
        \hline
        O2 & Product testing in a manufacturing production environment & O\\
        \hline
    \end{tabularx}
\end{table}
\section{Text structure}
\begin{description}
    \item[{\hyperref[cap:Introduction]{First chapter}}] 
    The first chapter introduce the company and its products and services.\\
    It also describes the idea of the project and the text structure.
    \item[{\hyperref[cap:process-methodologies]{Second chapter}}] The second chapter describes the process and the methodology used for the development of the project.
    \item[{\hyperref[cap:internship-desc]{Third chapter}}] Third chapter describes the internship experience, and how the project was planned and developed.
    \item[{\hyperref[cap:design-coding]{Fourth chapter}}] The fourth chapter describes how the project technologies were implemented for reach the objectives.
    \item[{\hyperref[cap:verification-validation]{Fifth chapter}}] The fifth chapter describes how the project was tested and validated.
    \item[{\hyperref[cap:conclusions]{Sixth chapter}}] The sixth chapter describes the conclusions of the project.
\end{description}

During the writing of this document, the following conventions were adopted:
\begin{itemize}
    \item acronyms, abbreviations and ambiguous or uncommon terms mentioned are defined in the glossary, located at the end of this document;
    \item for the first occurrence of the terms defined in the glossary, the following nomenclature is used: \emph{word}\glsfirstoccur;
	\item the terms that require further explanation like technical words are marked with a subscript \emph{italic};
\end{itemize}

% Variables
\newcommand{\myName}{Marco Bernardi}
\newcommand{\myTitle}{Generating Realistic Marble Textures using Generative Adversarial Networks}
\newcommand{\myDegree}{Tesi di laurea}
\newcommand{\myUni}{Università degli Studi di Padova}
\newcommand{\myFaculty}{Corso di Laurea in Informatica}
\newcommand{\myDepartment}{Dipartimento di Matematica ``Tullio Levi-Civita''}
\newcommand{\profTitle}{Prof.}
\newcommand{\myProf}{Lamberto Ballan}
\newcommand{\myLocation}{Padua}
\newcommand{\myAA}{2022-2023}
\newcommand{\myTime}{July 2023}

% PDF/A filecontents
\RequirePackage{filecontents}
\begin{filecontents*}{\jobname.xmpdata}
  \Title{Generating Realistic Marble Textures using Generative Adversarial Networks}
  \Author{Marco Bernardi}
  \Language{en-US}
  \Subject{This thesis aims to explore the use of Generative Adversarial Networks (GANs) to generate realistic marble textures. The main goal is to create a model that can generate marble textures with a high degree of realism and variability.}
  \Keywords{GANs\sep Marble\sep Texture\sep Deep Learning\sep Generative Adversarial Networks}
\end{filecontents*}

% Page format settings
% see: http://wwwcdf.pd.infn.it/AppuntiLinux/a2547.htm
\setlength{\parindent}{14pt}    % first row indentation
\setlength{\parskip}{0pt}       % paragraphs spacing

% Acronyms

\newacronym[description={\glslink{cncg}{Computerized Numerical Control}}]
    {cnc}{CNC}{Computerized Numerical Control}

\newacronym[description={\glslink{cadg}{Computer-aided Design}}]
    {cad}{CAD}{Computer-aided Design}

\newacronym[description={\glslink{iotg}{Internet of Things}}]
    {iot}{IOT}{Internet of Things}

\newacronym[description={\glslink{gang}{Generative Adversarial Network}}]
    {gan}{GAN}{Generative Adversarial Network}

\newacronym[description={\glslink{mlg}{Machine Learning}}]
    {ml}{ML}{Machine Learning}

\newacronym[description={\glslink{pocg}{Proof Of Concept}}]
    {poc}{POC}{Proof Of Concept}

\newacronym[description={\glslink{esrgang}{Enhanced Super Resolution Generative Adversarial Network}}]
    {esrgan}{ESRGAN}{Enhanced Super Resolution Generative Adversarial Network}

\newacronym[description={\glslink{apig}{Application Programming Interface}}]
    {api}{API}{Application Programming Interface}
    
\newacronym[description={\glslink{cgang}{Conditional Generative Adversarial Network}}]
    {cgan}{CGAN}{Conditional Generative Adversarial Network}

\newacronym[description={\glslink{fidg}{Fréchet Inception Distance}}]
    {fid}{FID}{Fréchet Inception Distance}

\newacronym[description={\glslink{ISG}{Inception Score}}]
    {is}{IS}{Inception Score}

\newacronym[description={\glslink{HEDG}{Holistically-Nested Edge Detection}}]
    {hed}{HED}{Holistically-Nested Edge Detection}

\newacronym[description={\glslink{CUDAG}{Compute Unified Device Architecture}}]
    {cuda}{CUDA}{Compute Unified Device Architecture}

\newacronym[description={\glslink{CuDNNG}{CUDA Deep Neural Network}}]
    {cudnn}{CuDNN}{CUDA Deep Neural Network}

\newacronym[description={\glslink{GPUG}{Graphics Processing Unit}}]
    {gpu}{GPU}{Graphics Processing Unit}

    \newacronym[description={\glslink{FAIRG}{Fair Artificial Intelligence Research}}]
    {fair}{FAIR}{Fair Artificial Intelligence Research}

\newacronym[description={\glslink{SCMG}{Source Control Management}}]
    {scm}{SCM}{Source Control Management}

\newacronym[description={\glslink{CPUG}{Central Processing Unit}}]
    {cpu}{CPU}{Central Processing Unit}
% Glossary entries
\newglossaryentry{apig}{
    name=\glslink{api}{API},
    text=API,
    sort=api,
    description={\emph{API, Application Programming Interface}. It is a set of clearly defined methods of communication between various software components}
}
\newglossaryentry{cncg} {
    name=\glslink{cnc}{CNC},
    text=CNC,
    sort=cnc,
    description={\emph{CNC, Computerized Numerical Control}. It is a computerized manufacturing process in which pre-programmed software and code controls the movement of production equipment}
}

\newglossaryentry{cadg} {
    name=\glslink{cad}{CAD},
    text=CAD,
    sort=cad,
    description={\emph{CAD, Computer-aided Design}. It is the use of computers to aid in the creation, modification, analysis, or optimization of a design}
}

\newglossaryentry{iotg} {
    name=\glslink{iot}{IOT},
    text=IOT,
    sort=iot,
    description={\emph{IOT, Internet of Things}. It is the network of physical devices, vehicles, home appliances, and other items embedded with electronics, software, sensors, actuators, and connectivity which enables these things to connect and exchange data}
}

\newglossaryentry{gang} {
    name=\glslink{gan}{GAN},
    text=GAN,
    sort=gan,
    description={\emph{GAN, Generative Adversarial Network}. It is a class of machine learning systems invented by Ian Goodfellow in 2014. Two neural networks contest with each other in a game. Given a training set, this technique learns to generate new data with the same statistics as the training set}
}

\newglossaryentry{mlg}{
    name=\glslink{ml}{ML},
    text=ML,
    sort=ml,
    description={\emph{ML, Machine Learning}. It is the study of computer algorithms that improve automatically through experience. It is seen as a subset of artificial intelligence}
}

\newglossaryentry{pocg}{
    name=\glslink{poc}{POC},
    text=POC,
    sort=poc,
    description={\emph{POC, Proof Of Concept}. It is a realization of a certain method or idea in order to demonstrate its feasibility, or a demonstration in principle with the aim of verifying that some concept or theory has practical potential}
}

\newglossaryentry{esrgang}{
    name=\glslink{esrgan}{ESRGAN},
    text=ESRGAN,
    sort=esrgan,
    description={\emph{ESRGAN, Enhanced Super Resolution Generative Adversarial Network}. It is a state-of-the-art method for generating realistic textures during image super-resolution}
}

\newglossaryentry{cgang}{
    name=\glslink{cgan}{CGAN},
    text=CGAN,
    sort=cgan,
    description={\emph{CGAN, Conditional Generative Adversarial Network}. It is a type of GAN that uses additional information to generate images}
}
\newglossaryentry{fidg}{
    name=\glslink{fid}{FID},
    text=FID,
    sort=fid,
    description={\emph{FID, Fréchet Inception Distance}. It is a metric used to evaluate the quality of generated images}
}
\newglossaryentry{ISG}{
    name=\glslink{is}{IS},
    text=IS,
    sort=is,
    description={\emph{IS, Inception Score}. It is a metric used to evaluate the quality of generated images}
}
\newglossaryentry{HEDG}{
    name=\glslink{hed}{HED},
    text=HED,
    sort=hed,
    description={\emph{HED, Holistically-Nested Edge Detection}. It is a state-of-the-art method for edge detection}
}
\newglossaryentry{CUDAG}{
    name=\glslink{cuda}{CUDA},
    text=CUDA,
    sort=cuda,
    description={\emph{CUDA, Compute Unified Device Architecture}. It is a parallel computing platform and application programming interface model created by Nvidia}
}
\newglossaryentry{CuDNNG}{
    name=\glslink{cudnn}{CuDNN},
    text=CuDNN,
    sort=cudnn,
    description={\emph{CuDNN, CUDA Deep Neural Network}. It is a GPU-accelerated library of primitives for deep neural networks}
}
\newglossaryentry{GPUG}{
    name=\glslink{gpu}{GPU},
    text=GPU,
    sort=gpu,
    description={\emph{GPU, Graphics Processing Unit}. It is a specialized electronic circuit designed to rapidly manipulate and alter memory to accelerate the creation of images in a frame buffer intended for output to a display device}
}
\newglossaryentry{FAIRG}{
    name=\glslink{fair}{FAIR},
    text=FAIR,
    sort=fair,
    description={\emph{FAIR, Fair Artificial Intelligence Research}. It is a research unit within Facebook that is focused on advancing the state of the art in AI}
}

\newglossaryentry{SCMG}{
    name=\glslink{scm}{SCM},
    text=SCM,
    sort=scm,
    description={\emph{SCM, Source Control Management}. It is the practice of tracking and managing changes to software code}
}

\newglossaryentry{CPUG}{
    name=\glslink{cpu}{CPU},
    text=CPU,
    sort=cpu,
    description={\emph{CPU, Central Processing Unit}. It is the electronic circuitry within a computer that executes instructions that make up a computer program}
}
\makeglossaries

\bibliography{appendix/bibliography}

\defbibheading{bibliography} {
    \cleardoublepage
    \phantomsection
    \addcontentsline{toc}{chapter}{\bibname}
    \chapter*{\bibname\markboth{\bibname}{\bibname}}
}

% Spacing between entries
\setlength\bibitemsep{1.5\itemsep}

\DeclareBibliographyCategory{opere}
\DeclareBibliographyCategory{web}

\addtocategory{opere}{womak:lean-thinking}
\addtocategory{web}{site:agile-manifesto}

\defbibheading{opere}{\section*{Riferimenti bibliografici}}
\defbibheading{web}{\section*{Siti Web consultati}}


\captionsetup{
    tableposition=top,
    figureposition=bottom,
    font=small,
    format=hang,
    labelfont=bf
}

% Images path
\graphicspath{{images/}}

\hypersetup{
    %hyperfootnotes=false,
    %pdfpagelabels,
    colorlinks=true,
    linktocpage=true,
    pdfstartpage=1,
    pdfstartview=,
    breaklinks=true,
    pdfpagemode=UseNone,
    pageanchor=true,
    pdfpagemode=UseOutlines,
    plainpages=false,
    bookmarksnumbered,
    bookmarksopen=true,
    bookmarksopenlevel=1,
    hypertexnames=true,
    pdfhighlight=/O,
    %nesting=true,
    %frenchlinks,
    urlcolor=webbrown,
    linkcolor=RoyalBlue,
    citecolor=webgreen
    %pagecolor=RoyalBlue,
}

% Delete all links and show them in black
\if \isprintable 1
    \hypersetup{draft}
\fi

% Itemize symbols
%\renewcommand{\labelitemi}{$\ast$}
%\renewcommand{\labelitemi}{$\bullet$}
%\renewcommand{\labelitemii}{$\cdot$}
%\renewcommand{\labelitemiii}{$\diamond$}
%\renewcommand{\labelitemiv}{$\ast$}

% Listings setup
\lstset{
    language=[LaTeX]Tex,%C++,
    keywordstyle=\color{RoyalBlue}, %\bfseries,
    basicstyle=\small\ttfamily,
    %identifierstyle=\color{NavyBlue},
    commentstyle=\color{Green}\ttfamily,
    stringstyle=\rmfamily,
    numbers=none, %left,%
    numberstyle=\scriptsize, %\tiny
    stepnumber=5,
    numbersep=8pt,
    showstringspaces=false,
    breaklines=true,
    frameround=ftff,
    frame=single
}

\definecolor{webgreen}{rgb}{0,.5,0}
\definecolor{webbrown}{rgb}{.6,0,0}

% \omiss produces '[...]'
\newcommand{\omissis}{[\dots\negthinspace]}

% Hyphenation rules
\hyphenation {
    ma-cro-istru-zio-ne
    gi-ral-din
}

\newcommand{\sectionname}{sezione}
\addto\captionsitalian{\renewcommand{\figurename}{Figura}
                       \renewcommand{\tablename}{Tabella}}

\newcommand{\glsfirstoccur}{\ap{{[g]}}}

\newcommand{\intro}[1]{\emph{\textsf{#1}}}

% Risks environment
\newcounter{riskcounter}                % define a counter
\setcounter{riskcounter}{0}             % set the counter to some initial value

%%%% Parameters
% #1: Title
\newenvironment{risk}[1]{
    \refstepcounter{riskcounter}        % increment counter
    \par \noindent                      % start new paragraph
    \textbf{\arabic{riskcounter}. #1}   % display the title before the content of the environment is displayed
}{
    \par\medskip
}

\newcommand{\riskname}{Rischio}

\newcommand{\riskdescription}[1]{\textbf{\\Descrizione:} #1.}

\newcommand{\risksolution}[1]{\textbf{\\Soluzione:} #1.}

% Use case environment
\newcounter{usecasecounter}             % define a counter
\setcounter{usecasecounter}{0}          % set the counter to some initial value

%%%% Parameters
% #1: ID
% #2: Nome
\newenvironment{usecase}[2]{
    \renewcommand{\theusecasecounter}{\usecasename #1}  % this is where the display of
                                                        % the counter is overwritten/modified
    \refstepcounter{usecasecounter}             % increment counter
    \vspace{10pt}
    \par \noindent                              % start new paragraph
    {\large \textbf{\usecasename #1: #2}}       % display the title before the
                                                % content of the environment is displayed
    \medskip
}{
    \medskip
}

\newcommand{\usecasename}{UC}

\newcommand{\usecaseactors}[1]{\textbf{\\Attori Principali:} #1. \vspace{4pt}}
\newcommand{\usecasepre}[1]{\textbf{\\Precondizioni:} #1. \vspace{4pt}}
\newcommand{\usecasedesc}[1]{\textbf{\\Descrizione:} #1. \vspace{4pt}}
\newcommand{\usecasepost}[1]{\textbf{\\Postcondizioni:} #1. \vspace{4pt}}
\newcommand{\usecasealt}[1]{\textbf{\\Scenario Alternativo:} #1. \vspace{4pt}}

% Namespace description environment
\newenvironment{namespacedesc}{
    \vspace{10pt}
    \par \noindent  % start new paragraph
    \begin{description}
}{
    \end{description}
    \medskip
}

\newcommand{\classdesc}[2]{\item[\textbf{#1:}] #2}
